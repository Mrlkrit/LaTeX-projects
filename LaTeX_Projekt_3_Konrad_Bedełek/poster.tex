
\documentclass[final]{beamer}

\usepackage[orientation=landscape,size=a3]{beamerposter}%size modify
\usepackage{multicol}%for columns
\usepackage[utf8]{inputenc}
\usepackage{color}%for colors
\usepackage{tikz}%draw
\usepackage{graphicx}% images

% Colors
\definecolor{marronrp3}{rgb}{.9,.9,.7}
\definecolor{salmon}{rgb}{1,.9,.8}





% ===========================================================================

\begin{document}
\begin{frame}

\begin{center}
  \begin{minipage}{.19\linewidth}
    \includegraphics[width=.6\linewidth]{logo.jpg}
  \end{minipage}
  %Cool title
  \begin{minipage}{.6\linewidth}
    \begin{center}
      \Huge \textbf{A cool title for this LaTeX project}
    \end{center}
  \end{minipage}
  %&
  %Author,university and thanks
  \hspace{.03\linewidth}
  \begin{minipage}{0.16\linewidth}
    \begin{flushright}
      Konrad Bedełek\\
      Dept. Mechatronics\\
      Warsaw University of Technology, Warsaw\\
      \vspace{.1cm} \small \emph{These results were obtained under the
        supervision and help of Krzysztof Wildner. I wish to thank
        him for his explanations and suggestions.}
    \end{flushright}
  \end{minipage}
\end{center}

\vspace{.1cm}

% ---------------------------------------------------------------------------

\setlength{\columnsep}{1cm}
\begin{multicols}{3}
%%%%%%%%%%%%%%%%%%%%%%%%%%%%%%%%%%%%%%%
%some kind of abstract in color
\noindent
\fcolorbox{black}{salmon}{
  \begin{minipage}[t]{.96\linewidth}
    \vspace{.05cm}
    \begin{center}
      \vspace{.1cm}
      \section*{\Huge Main result}
	
      \Large In \cite{kowal1} we prove that the well-known
      asymptotic behavior of students to the generalized
      university system takes place for general initial data,
      extending the previous knowledge that placed some restrictions
      on it.
    \end{center}
  \end{minipage}
}
\vspace{.3cm}
%%%%%%%%%%%%%%%%%%%%%%%%%%%%%%%%%%%%%%



%%%%%%%%%%%%%%%%%%%%%%%%%%%%%%%%%%%%%%
%text
The graduation equations describe the
evolution of a large number of students which can stick together or
break. Here we deal with the discrete version.
%%%%%%%%%%%%%%%%%%%%%%%%%%%%%%%%%%%%%%
%import images and make subtitles using tables
\begin{center}

  \begin{tabular}[h]{lll}
    $c_j$  &  $\equiv$  &  graduates
  \end{tabular}

  \vspace{.3cm}


  \begin{minipage}[t]{.4\linewidth}
    \begin{center}

      \includegraphics[width=3cm]{logo1.png}


      \begin{tabular}{cc}
        $b_{jk}$  $\equiv$ &  rate of occurrence of\\
        &  enlightenment $j+k \to j,k$
      \end{tabular}
    \end{center}
  \end{minipage}
  
  \hspace{.3cm}

  \begin{minipage}[t]{.4\linewidth}
    \begin{center}

      \includegraphics[width=3cm]{logo3.jpg}


      \begin{tabular}{cc}
        $a_{jk}$   $\equiv$ &  rate of occurrence of\\
        &  goofiness $j \to j+k$
      \end{tabular}
    \end{center}                    
  \end{minipage}

\end{center}

%%%%%%%%%%%%%%%%%%%%%%%%%%%%%%%%%%%%%%
% cool ecquation
\vspace{.4cm}

\noindent
\colorbox{marronrp3}{
  \begin{minipage}[t]{.96\linewidth}
    \Large
    \begin{align*}
      \frac{d}{dt} c_j
      & = &&  \frac{1}{2} \sum_{k=1}^{j-1} a_{k,j-k}  c_k c_{j-k}
      &  \text{Students entering university}\\
      && - &\sum_{k=1}^{\infty} a_{jk} c_j c_k
      &   \text{Students that run away after one week}\\
      && + &\sum_{k=j+1}^{\infty} b_{j,k-j} c_k
      &   \text{Students who try once again}\\
      && - &\frac{1}{2} \sum_{k=1}^{j-1} b_{k,j-k} c_j
      & \text{Students that do not survive finals}
    \end{align*}
    \vspace{.02cm}
  \end{minipage}
}

\vspace{.3cm}
%%%%%%%%%%%%%%%%%%%%%%%%%%%%%%%%%%%%%%
\columnbreak
% some text and draw star using tikz
Star is any massive self-luminous celestial body of gas that shines by radiation derived from its internal energy sources.
Of the tens of billions of trillions of stars composing the observable universe, only a very small percentage are visible to the naked eye.
Many stars occur in pairs, multiple systems, or star clusters.
The members of such stellar groups are physically related through common origin and are bound by mutual gravitational attraction.
Somewhat related to star clusters are stellar associations,
which consist of loose groups of physically similar stars that have insufficient mass as a group to remain together as an organization.
So should you be worried about the star?

\begin{figure}
   \centering
      \begin{tikzpicture}[scale=4]
         \draw[step=0.25cm,color=gray] (-1,-1) grid (1,1);
         \draw (1,0) -- (0.2,0.2) -- (0,1) -- (-0.2,0.2) -- (-1,0)
          -- (-0.2,-0.2) -- (0,-1) -- (0.2,-0.2) -- cycle;
      \end{tikzpicture}
   \caption{It's my big star}

\end{figure}
%%%%%%%%%%%%%%%%%%%%%%%%%%%%%%%%%%%%%%

%%%%%%%%%%%%%%%%%%%%%%%%%%%%%%%%%%%%%%
%more colored blocks
\noindent
\begin{center}
  \noindent
  \colorbox{marronrp3}{
    \begin{minipage}[t]{.96\linewidth}
      \begin{align*}
        & \text{\Large Below critical mass}
        &\to \quad
        &\begin{cases}
          \text{ \Large No danger }\\
          \text{ \Large No need to worry }
        \end{cases}
        &
        \\
        &\text{\Large Over critical mass }
        &\to \quad
        &\begin{cases}
          \text{ \Large Run!!!}\\
          \text{\Large  Too late }
        \end{cases}
        &
      \end{align*}
    \end{minipage}
  }
\end{center}
%%%%%%%%%%%%%%%%%%%%%%%%%%%%%%%%%%%%%%
%table

\begin{center}  
\vspace{.5cm}

\Large
\begin{tabular}[t]{c|c}
  \multicolumn{2}{c}{\huge \textbf{The Stars}}
  \vspace{.3cm}
  \\
  VY Canis Majoris& Big Boi\\
  \hline
  LGGS &Even\\
  J004520.67+414717.3 &BIGGER\\
  \hline
  Stephenson & That is \\
  2-18 & quite big
\end{tabular}

\end{center}


%%%%%%%%%%%%%%%%%%%%%%%%%%%%%%%%%%%%%%
  
\columnbreak
%%%%%%%%%%%%%%%%%%%%%%%%%%%%%%%%%%%%%%
Lorem ipsum dolor sit amet, consectetur adipiscing elit. Duis et quam nisi. Donec finibus ultrices sem, quis feugiat neque luctus eget. Integer tincidunt luctus lectus, a vehicula justo facilisis sed. 
Suspendisse egestas arcu sed ante finibus egestas. Nullam viverra orci neque, at porttitor tortor fermentum non. Lorem ipsum dolor sit amet, consectetur adipiscing elit. 
Duis rutrum fringilla libero sed lobortis. Nunc odio eros, dignissim vitae tristique at, tincidunt non lacus. Praesent euismod lobortis blandit.
Proin pellentesque libero lorem, eleifend semper nulla maximus non.\cite{kowal2}

Fusce at ligula velit. Phasellus fringilla placerat fringilla. Integer porta ligula nunc. Integer nec arcu suscipit tellus tincidunt iaculis quis non urna.
Sed dignissim tristique elit vitae volutpat. Cras vel mi ac eros porttitor pellentesque at eget ipsum. Aliquam nisl tortor, vestibulum ac vehicula eu, pulvinar nec quam. 
Praesent sit amet diam id dolor accumsan consectetur at eget sem. Nulla ac lorem et mauris consequat laoreet. Etiam elit neque, mollis nec accumsan nec, ultricies vel enim. 
Etiam non fermentum sapien, ut eleifend ex. Etiam iaculis auctor dui. Sed efficitur ultrices viverra. In volutpat placerat bibendum.\cite{hans1}

%a bunch of text and simple list


\begin{table}
{\LARGE{List of steps needed to be taken}}
  \begin{enumerate}
    \item Do A.
    \item Do B.
    \item Do C.
  \end{enumerate}
\end{table}
The purpose of lorem ipsum is to create a natural looking block of text (sentence, paragraph, page, etc.) that doesn't distract from the layout.
A practice not without controversy, laying out pages with meaningless filler text can be very useful when the focus is meant to be on design, not content.
\vspace{.5cm}

The passage experienced a surge in popularity during the 1960s when Letraset used it on their dry-transfer sheets,
and again during the 90s as desktop publishers bundled the text with their software. 
Today it's seen all around the web; on templates, websites, and stock designs. 
Use our generator to get your own, or read on for the authoritative history of lorem ipsum.
\vspace{.4cm}
%%%%%%%%%%%%%%%%%%%%%%%%%%%%%%%%%%%%%%

\small
%bibliografia
\begin{thebibliography}{9}
\bibitem{kowal1} Kowalski A. How I met your mother, Warszawa, 2018
\bibitem{kowal2} Kowalski J. I don't know what I did, 2015
\bibitem{hans1} Copernicus H. Mein Land, 1939
\end{thebibliography}


\end{multicols}
\end{frame}
\end{document}