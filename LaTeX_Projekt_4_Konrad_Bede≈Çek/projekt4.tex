\documentclass{beamer}%klasa beamer
\usepackage[T1]{fontenc}%to jest tu potrzebne bo inaczej coś tex protestuje

%do slajdu tytułowego
\title{Prezentacja \LaTeX}
\subtitle{Tworzenie tabel}
\author{Konrad Bedełek}
\institute{Wydział Mechatroniki\\ Politechnika Warszawska}
\date{\today}

\usetheme{Warsaw}%motyw
\usecolortheme{rose}%motyw kolorystyczny



\begin{document}

%Pierwszy slajd(tytułowy)
\begin{frame}
\titlepage
\end{frame}

%Drugi slajd
\begin{frame}

\frametitle{Tworzenie tabel}

Do tworzenia prostych tabel w Beamerze służy otoczenie tabular.
\begin{itemize}[<+->]%dołączanie elementów 
\item Tabele zaczynają się poleceniem \textbackslash begin\{tabular\}\{ccc\}.
\item \{ccc\} określa liczbę kolumn oraz wyrównanie każdej z nich.\\
Ta tabela ma trzy kolumny, każda wyrównana centralnie.
\item Kolumny mogą być wyrównywane na lewo \{l\}, centralnie\\
\{c\}, lub na prawo \{r\}.
\item Można mieszać wyrównywania. Np. \{lcrrr\}.
\item Tabele są tworzone wiersz po wierszu. Znak \& rozdziela\\
komórki, a każdy wiersz jest zakończony przez \textbackslash\textbackslash.
\item \textbackslash end\{tabular\} zamyka tabelę.

\end{itemize}
\end{frame}

%Trzeci slajd
\begin{frame}
\frametitle{Tworzenie tabel}
Typowa tabela w Beamerze wygląda następująco:

\begin{block}{Przykładowa tabela}%blok informacyjny z zapisanymi komendami
\textbackslash begin\{tabular\}\{ccc\}\\
cell 1 \& cell 2 \& cell 3 \\
cell 4 \& cell 5 \& cell 6 \\
\textbackslash end\{tabular\}

\end{block}
\pause%w następnym slajdzie dołącz tabelę
\begin{tabular}{ccc}
cell 1 & cell 2 & cell 3 \\
cell 4 & cell 5 & cell 6 \\
\end{tabular}

\end{frame}

%Czwarty slajd

\begin{frame}
\frametitle{Tworzenie tabel}
Możemy dodać znaki | pomiędzy kolumnami, aby je rozdzielić
pionowymi kreskami.
\begin{block}{Przykładowa tabela}
\textbackslash begin\{tabular\}\{|c|c|c|\}\\
cell 1 \& cell 2 \& cell 3 \\
cell 4 \& cell 5 \& cell 6 \\
\textbackslash end\{tabular\}

\end{block}
\pause
\begin{tabular}{|c|c|c|}
cell 1 & cell 2 & cell 3 \\
cell 4 & cell 5 & cell 6 \\
\end{tabular}

\end{frame}



%Piąty slajd

\begin{frame}
\frametitle{Tworzenie tabel}
Aby stworzyć nagłówki możemy użyć \textbackslash textbf i \textbackslash hline:
\begin{block}{Przykładowa tabela}
\textbackslash begin\{tabular\}\{c||c|c|c|\}\\
\& \textbackslash textbf\{Ng. 1\}\& \textbackslash textbf\{Ng. 2\}\& \textbackslash textbf\{Ng. 3\}\\
\textbackslash hline \textbackslash hline\\
\textbackslash textbf\{Ng. 4\} \& cell 1 \& cell 2 \& cell 3 \\
\textbackslash hline\\
\textbackslash textbf\{Ng. 5\} \& cell 4 \& cell 5 \& cell 6 \\
\textbackslash end\{tabular\}


\end{block}
\pause
\begin{tabular}{c||c|c|c|}
& \textbf{Ng. 1}& \textbf{Ng. 2}& \textbf{Ng. 3}\\
\hline \hline
\textbf{Ng. 4} & cell 1 & cell 2 & cell 3 \\
\hline
\textbf{Ng. 5} & cell 4 & cell 5 & cell 6 \\
\end{tabular}


\end{frame}

%bibliografia
\begin{frame}
\frametitle{Literatura}
\begin{thebibliography}{9}
\bibitem{wyklad} Wildner K. Slajdy z wykładu: Dokumentacja i prezentacja wyników badań i projektów z
zastosowaniem środowiska \LaTeX

\end{thebibliography}
\end{frame}

\end{document}